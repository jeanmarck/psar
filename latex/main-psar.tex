\documentclass[12pt,envcountsame]{article}


%\usepackage{epsf}
%\newcommand{\epsf}[1]{\epsfbox{#1}}

\usepackage{latexsym}
\usepackage{amssymb,amsmath}
\usepackage{amsthm,amsfonts}
\usepackage{gastex}
\usepackage[dvips]{graphics}
\usepackage{pstricks}
\usepackage{a4wide}
\usepackage{multicol}
\usepackage{pst-tree}

\usepackage[latin1]{inputenc}
%\usepackage[T1]{fontenc}
\usepackage[french]{babel}

\setcounter{tocdepth}{2}

\newcommand{\Q}{\mathbb{Q}}
\newcommand{\R}{\mathbb{R}}
\newcommand{\N}{\mathbb{N}}
\newcommand{\Z}{\mathbb{Z}}
\newcommand{\T}{\mathbb{T}}
\newcommand{\Bo}{\mathbb{B}}

\newcommand{\Au}{\mathcal{A}}
\newcommand{\B}{\mathcal{B}}
\newcommand{\C}{\mathcal{C}}
\newcommand{\D}{\mathcal{D}}
\newcommand{\F}{\mathcal{F}}
\newcommand{\G}{\mathcal{G}}
\newcommand{\K}{\mathcal{K}}
\newcommand{\Hr}{\mathcal{H}}
\newcommand{\I}{\mathcal{I}}
\newcommand{\La}{\mathcal{L}}
\newcommand{\M}{\mathcal{M}}
\newcommand{\Net}{\mathcal{N}}
\newcommand{\Pa}{\mathcal{P}}
\newcommand{\Rel}{\mathcal{R}}
\newcommand{\Sy}{\mathcal{S}}
\newcommand{\Ts}{\mathcal{T}}
\newcommand{\X}{\mathcal{X}}

\newcommand{\rel}{\bowtie}
\newcommand{\tr}{\xrightarrow}
\newcommand{\opeq}{\leftrightarrow}
\newcommand{\fee}{\varphi}
\newcommand{\eps}{\varepsilon}
\newcommand{\vect}[1]{\mathbf{#1}}
\newcommand{\fut}{\overrightarrow}
\newcommand\sui[1][a]{\ensuremath{\left(#1_n\right)_{n\in \N}}}

\newcommand{\E}{\textsf{E}}
\newcommand{\A}{\textsf{A}}

\DeclareMathOperator{\cla}{cl}

\theoremstyle{plain}

\newtheorem{theorem}{Th�or�me}
\newtheorem{lemma}[theorem]{Lemme}
\newtheorem{corollary}[theorem]{Corollaire}
\newtheorem{proposition}[theorem]{Proposition}

\newtheorem{definition}[theorem]{D�finition}

\theoremstyle{remark}
\newtheorem{remark}[theorem]{Remarque}
\newtheorem{example}[theorem]{Exemple}

\begin{document}
\thispagestyle{empty}
\noindent UPMC, 2I005 \hfill Ann�e 2015--2016 \\
B�atrice B�rard  \hfill Aide m�moire\\


%\iffalse
%\vspace*{6cm}


\begin{center}
  {\large \bf Structures discr�tes : logique}  
\end{center}

\tableofcontents

%\vfill 
%\fi
%\input{ens-rel0}
%\input{rel-ordre}
%\input{induction}
%\input{auto}
\clearpage
\section{Fonctions bool�ennes}
\subsection{Alg�bre de Boole}
\begin{definition}[Alg�bre de Boole]
  Une alg�bre de Boole est un tuple
  
  Le projet  \textbf{ "Interface graphique pour la Logique en L3"}consiste � d�velopper un outil dynamique et robuste pour am�liorer l'enseignement  de la logique en Licence 3. Ce projet nous a �t� soumis dans le cadre de L'UE PSAR du master 1 Informatique sp�cialit� SAR.
Il est sous la responsabilit� de Mr Fabrice Kordon suivit par Mme B�atrice Berard,Mr Mathieu Jaume et Mme B�n�dicte Legastelois.
  $\mathcal{B}=(E,\,\bot,\,\top,\,\vee,\,\wedge,\,\bar{\,})$ o� $E$
  est un ensemble, $\bot$ et $\top$ sont deux �l�ments distincts de
  $E$, $\vee$ et $\wedge$ sont deux op�rations binaires, $\bar{\,}$
  est une op�ration unaire, satisfaisant les propri�t�s suivantes :
\begin{itemize}
\item \emph{Associativit�} : pour tous $ a,\, b,\, c\in E$, 
$(a\vee b)\vee c=a\vee(b\vee c)$ et $(a\wedge b)\wedge c=a\wedge(b\wedge c)$
\item \emph{Commutativit�} : pour tous $a,\, b\in E$, 
$a\vee b=b\vee a$ et $a\wedge b=b\wedge a$
\item \emph{Distributivit�} d'une loi par rapport � l'autre: 
pour tous $a,\, b,\, c\in E$, \\
  $(a\vee b)\wedge c=(a\wedge c)\vee(b\wedge c)$ et $(a\wedge b)\vee
  c=(a\vee c)\wedge(b\vee c)$
\item \emph{Absorption} : pour tous $a,\, b\in E$, 
$a\wedge(a\vee b)=a$ et $a\vee(a\wedge b)=a$
\item \emph{Idempotence} : pour tout $a\in E$, 
$a\vee a=a$ et $a\wedge a=a$
\item \emph{Bornes} : pour tout $a\in E$, 
$a\wedge\bot=\bot,\ \  a\vee\bot=a$ et $a\wedge\top=a,\ \  a\vee\top=\top$
\item \emph{Compl�mentarit�} : pour tout $a\in E$, 
 $a\wedge\bar{a}=\bot$ et $a\vee\bar{a}=\top$
\end{itemize}
\end{definition}
\medskip{}

\noindent \textbf{Exemples.}
\begin{enumerate}
\item Soit $E=\mathcal{P}(A)$ pour un ensemble $A$ non vide.
On d�finit alors une alg�bre de Boole avec : \\

\begin{tabular}{ccccc}
\hline 
$\bot$ & $\top$ & $\vee$ & $\wedge$ & $\bar{\,}$\tabularnewline
\hline 
$\textrm{�}$ & A & $\cup$ & $\cap$ & compl�mentaire\tabularnewline
\hline
\end{tabular}

\bigskip{}


\item Soit $\Bo=\{ 0,\,1 \} $. On d�finit alors une alg�bre
de Boole avec : \\

{}\begin{tabular}{ccccc}
\hline 
$\bot$ & $\top$ & %
$\vee$
 & %
$\wedge$
 & %
$\bar{\,}$
\tabularnewline
\hline 
0 (faux) & 1 (vrai) & ou (disjonction) & et (conjonction) & 
n�gation\tabularnewline
\hline
\end{tabular}

\medskip \noindent Pour les calculs dans $\Bo$, on note g�n�ralement
$+$ pour $\vee$ et $.$ ou rien pour $\wedge$.
\end{enumerate}

\subsection{Fonctions bool�ennes}
\begin{definition}[Fonction bool�enne]
Soit $n\in\N$. Une fonction bool�enne � $n$ arguments
est une application $f\colon\Bo^{n}\longrightarrow \Bo$.
\end{definition}
\noindent \textbf{Remarque.}
Si $n=0$, il y a deux fonctions constantes : $0,\,1$.\\
Si $n=1$, il y a quatre fonctions : $x \rightarrow 0,\, x \rightarrow 1,\,
x \rightarrow x$ et $x \rightarrow\bar{x}$.\\
Il y a $2^{2^{n}}$ fonctions bool�ennes � $n$ arguments.

%\vspace{1in}
\noindent \textbf{Exemple.}
Table de v�rit� de la fonction $NAND$ (� deux arguments), d�finie 
par :\\ $NAND(x,y) = \overline{x.y}$.\\

% \begin{multicols}{2}

%   \begin{pspicture}(0,-0.551632)(3.74,0.548368)
%     \usefont{T1}{ptm}{m}{n} \rput(0.28,0.273368){$x$}
%     \psline[linewidth=0.04cm](0.44,0.26336798)(1.0410175,0.26336798)
%     \psline[linewidth=0.04cm](0.44,-0.27988198)(1.0410175,-0.27988198)
%     \psline[linewidth=0.04cm](2.278605,-0.0016320016)(2.8796225,-0.0016320016)
%     \psline[linewidth=0.04cm](1.0289972,0.515118)(1.0410175,-0.531632)
%     \rput{-88.68047}(1.4399921,1.475771){\psarc[linewidth=0.04]
%     (1.4750738,0.0011139058){0.51871073}{-1.3322198}{178.06238}}
%     \psline[linewidth=0.04cm](1.5098113,-0.518382)(1.0530379,-0.518382)
%     \psline[linewidth=0.04cm](1.5098113,0.528368)(1.0169768,0.528368)
%     \usefont{T1}{ptm}{m}{n} \rput(0.27,-0.24663201){$y$}
%     \usefont{T1}{ptm}{m}{n} \rput(3.3,0.013367998){$\overline{x.y}$}
%     \pscircle[linewidth=0.04,dimen=outer](2.15,-0.0016320094){0.15} 
%\end{pspicture}

\begin{tabular}{|c|c|c|}
\hline 
$x$ & $y$ & $NAND(x,\, y)$\tabularnewline
\hline
\hline 
0 & 0 & 1\tabularnewline
\hline 
0 & 1 & 1\tabularnewline
\hline 
1 & 0 & 1\tabularnewline
\hline 
1 & 1 & 0\tabularnewline
\hline
\end{tabular}

%\end{multicols}

\begin{theorem}\label{th:poly}
  Toute fonction bool�enne $f$ � $n$ arguments, $(n\geqslant1)$,
  s'�crit comme combinaison de ses arguments ou de leurs
  compl�mentaires avec somme et produit.
\end{theorem}

\noindent \textbf{Exemple.}
$NAND(x,\, y)=\bar{x}\cdot\bar{y}+\bar{x}\cdot y+x\cdot\bar{y}$


\medskip \noindent Ce th�or�me se d�montre par r�currence
sur $n$, en utilisant le lemme suivant~:
\begin{lemma}
Soit $f$ une fonction bool�enne � $n$ arguments.\\
Alors $f(x_{1},\,\dots,\, x_{n})=x_{1}f(1,\, x_{2},\,\dots,\, x_{n})+\overline{x_{1}}f(0,\, x_{2},\,\dots,\, x_{n})$
\end{lemma}
\begin{proof}
  Posons $g(x_{1},\,\dots,\, x_{n})=x_{1}f(1,\, x_{2},\,\dots,\,
  x_{n})+\overline{x_{1}}f(0,\, x_{2},\,\dots,\, x_{n})$ et montrons
  que $f=g$, c'est-�-dire que pour tout $n$-uplet $(x_{1},\,\dots,\,
  x_{n})$, $g(x_{1},\,\dots,\, x_{n})=f(x_{1},\,\dots,\, x_{n})$.
\begin{itemize}
\item Si $x_{1}=1,\; g(x_{1},\,\dots,\, x_{n})=f(1,\, x_{2},\,\dots,\,
  x_{n})=f(x_{1},\,\dots,\, x_{n})$
\item Si $x_{1}=0,\, g(x_{1},\,\dots,\, x_{n})=f(0,\, x_{2},\,\dots,\,
  x_{n})=f(x_{1},\,\dots,\, x_{n})$
\end{itemize}
Donc $f=g$.
\end{proof}

\subsection{Formes normales}
Pour une fonction bool�enne $f$ � $n$ arguments, on note $b=
(b_{1},\,\dots,\, b_{n})$ un �l�ment de $\Bo^{n}$ et
$\mathcal{D}_{f}=\{ b \in \Bo^{n}/f(b)=1\} $.

\begin{definition}[Formes disjonctives et conjonctives]~\\
\vspace{-0.4cm}
\begin{itemize}
\item Une fonction est sous forme normale disjonctive (FND) si elle s'�crit
  comme une somme de produits de $x_{i}$ ou $\overline{x_{i}}$.
\item Une fonction est sous forme normale conjonctive (FNC) si elle s'�crit
  comme produit de sommes de $x_{i}$ ou $\overline{x_{i}}$.
\end{itemize}
\end{definition}

Pour une fonction bool�enne $f$ � $n$ arguments, une forme normale
disjonctive pour $f$ est obtenue par :
\[ f(x_{1},\,\dots,\,
x_{n})=\sum_{b\in\mathcal{D}_{f}}M_b(x_{1},\,\dots,\, x_{n})\mbox{ o�
} 
\begin{array}[t]{l}
  M_{b}(x_{1},\,\dots,\, x_{n})=x_{1}'\cdot\cdots\cdot x_{n}'\\
  \mbox{avec } x_{i}'=\begin{cases}
    x_{i} & \mbox{si }b_{i}=1\\
    \overline{x_{i}} & \mbox{si }b_{i}=0\end{cases}\end{array}\]


\noindent \textbf{Exemple.}  Pour la fonction $NAND$, on a :
$NAND(x,\, y)=\underbrace{\bar{x}\cdot\bar{y}}_{M_{(0,\,0)}(x,\,
  y)}+\underbrace{\bar{x}\cdot y}_{M_{(0,\,1)}(x,\,
  y)}+\underbrace{x\cdot\bar{y}}_{M_{(1,\,0)}(x,\, y)}$


Soit $f$ une fonction bool�enne � $n$ arguments, une forme normale
conjonctive pour $f$ est obtenue par : 
\[ f(x_{1},\,\dots,\,
x_{n})=\prod_{b\not \in \mathcal{D}_{f}}S_{b}(x_{1},\,\dots,\, x_{n})\mbox{
  o� }\begin{array}[t]{l}
  S_{b}(x_{1},\,\dots,\, x_{n})=x_{1}'+\cdots+x_{n}'\\
  \mbox{avec } x_{i}'=\begin{cases}
    x_{i} & \mbox{si }b_{i}=0\\
    \overline{x_{i}} & \mbox{si }b_{i}=1\end{cases}
\end{array}\]

\noindent \textbf{Exemple.}  Pour la fonction $NAND$, on a :
$NAND(x,\, y)=\underbrace{\bar{x}+\bar{y}}_{S_{(1,\,1)}(x,\, y)}$


%\clearpage
\section{Calcul propositionnel}
\subsection{Syntaxe}
\begin{definition}
  Soit $\mathcal{P}$ un ensemble de symboles propositionnels (ou
  variables propositionnelles). Les formules du calcul propositionnel
  sont d�finies inductivement par :
\begin{itemize}
\item[(B)] Si $p\in\mathcal{P}$, alors $p$ est une formule.
\item[(I)] Si $F$ est une formule, alors $\neg F$ est une formule,\\
  si $F_{1}$ et $F_{2}$ sont deux formules, alors $(F_{1}\vee F_{2})$
  et $(F_{1}\wedge F_{2})$ sont aussi des formules.
\end{itemize}
\end{definition}
\noindent \textbf{Exemple 1.}  $F=((q \wedge r) \vee \neg p))$ est une formule
utilisant les symboles $p$, $q$ et $r$.

\begin{definition}
  On d�finit deux nouvelles op�rations $\to$ et $\opeq$ par~:
\begin{itemize}
\item $F_{1}\to F_{2}= F_{2}\vee \neg F_{1}$
\item $F_{1}\opeq F_{2}=(F_{1}\to F_{2})\wedge(F_{2}\to F_{1})$
\end{itemize}
\end{definition}

\noindent \textbf{Exemple 2.}  $G=(p\to(p\to q))$ est une formule
utilisant les deux symboles $p$ et $q$.

\medskip \noindent \textbf{Remarque 1.}  Le symbole $\supset$ est
parfois utilis� au lieu de $\to$ pour l'implication et parfois aussi
$\equiv$ au lieu de $\opeq$.

\noindent \textbf{Remarque 2.} Les formules du calcul propositionnel
peuvent �tre vues comme les termes construits avec $F_0= \Pa$,
$F_1=\{\neg\}$ et $F_2 = \{\vee, \wedge, \to, \opeq \}$, avec une
notation infixe pour les op�rateurs binaires, c'est-�-dire par exemple
$p \vee q$ au lieu de $\vee(p,q)$.

\noindent \textbf{Exemple 3.}  $F=\neg p\wedge((q\vee r)\to s)$ est une
formule sur le sous-ensemble $\{ p,\, q,\, r,\, s\}$ de  $\mathcal{P}$, qui 
peut �tre repr�sent�e (comme un terme) par un arbre.

%\noindent 
% \begin{center}
%   \begin{pspicture}(0,-2.32)(3.38,2.32)
%     \definecolor{color58}{rgb}{0.6,0.0,0.6} \usefont{T1}{ptm}{m}{n}
%     \rput(1.5,2.125){$\wedge$} \usefont{T1}{ptm}{m}{n}
%     \rput(0.49,0.725){$\neg$} \usefont{T1}{ptm}{m}{n}
%     \rput(0.49,-0.675){$p$} \usefont{T1}{ptm}{m}{n}
%     \rput(1.9,-0.695){$\vee$} \usefont{T1}{ptm}{m}{n}
%     \rput(2.49,0.705){$\to$} \usefont{T1}{ptm}{m}{n}
%     \rput(1.29,-2.095){$q$} \usefont{T1}{ptm}{m}{n}
%     \rput(3.1,-0.695){$s$} \usefont{T1}{ptm}{m}{n}
%     \rput(2.51,-2.115){$r$}
%     \psline[linewidth=0.04cm,linecolor=color58](1.5,1.86)(2.5,0.96)
%     \psline[linewidth=0.04cm,linecolor=color58](0.5,0.96)(1.5,1.86)
%     \psline[linewidth=0.04cm,linecolor=color58](2.5,0.46)(3.1,-0.44)
%     \psline[linewidth=0.04cm,linecolor=color58](1.9,-0.44)(2.5,0.46)
%     \psline[linewidth=0.04cm,linecolor=color58](0.5,0.46)(0.5,-0.44)
%     \psline[linewidth=0.04cm,linecolor=color58](1.3,-1.84)(1.9,-0.94)
%     \psline[linewidth=0.04cm,linecolor=color58](1.9,-0.94)(2.5,-1.84)
% \end{pspicture}
% \end{center}


\subsection[S�mantique]{S�mantique : interpr�tation des formules}
\noindent \textbf{Exemple.}  On reprend l'exemple ci-dessus.  En
associant des valeurs dans $\Bo=\{ 0,\,1\}$ aux propositions $p$, $q$,
$r$ et $s$, on peut obtenir une valeur (�galement dans $\Bo$) pour la
formule $F$, et plus g�n�ralement pour toute formule portant sur des
propositions de $\{ p,\, q,\, r,\, s\}$. Par exemple, l'interpr�tation
$p \mapsto 0,\; q \mapsto 0,\; r \mapsto 1,\; s \mapsto 1$ produit la
valeur $1$ pour $F$. 

\medskip Dans le cas g�n�ral, une interpr�tation est une application
$I\colon\mathcal{P}\longrightarrow \Bo$. A partir d'une telle
application, qui associe � chaque proposition de $\mathcal{P}$ une
valeur dans $\Bo=\{ 0,\,1\}$, il est possible de d�duire une
interpr�tation de toutes les formules de CP.
\begin{definition}
  Soit $I\colon\mathcal{P} \longrightarrow \{ 0,\,1\}$ une
  interpr�tation des symboles de $\mathcal{P}$.  Le prolongement de
  $I$ aux formules du calcul propositionnel est l'application encore
  not�e $I$ (au lieu de $I^{*}$) de CP $\longrightarrow \{ 0,\,1\}$
  d�finie inductivement par :
\begin{itemize}
\item[(B)] Si $F=p$, alors $I(F)=I(p)$.
\item[(I)] Si $F=\neg G$, alors $I(F)=\overline{I(G)}$,\\
  si $F=F_{1}\vee F_{2}$ alors $I(F)=I(F_{1})+I(F_{2})$ \\ et si
  $F=F_{1}\wedge F_{2}$ alors $I(F)=I(F_{1})I(F_{2})$.
\end{itemize}
\end{definition}
\noindent Ceci correspond bien � interpr�ter les symboles d'op�rations
de la fa�on usuelle : $\neg$ comme la n�gation, $\wedge$ comme la
conjonction et $\vee$ comme la disjonction.
\begin{proposition}
Soit $I$ une interpr�tation. Alors : 
\begin{enumerate}
\item $I(F_{1}\to F_{2})=I(F_{2}) + \overline{I(F_1)}$
\item $I(F_{1}\opeq F_{2})=I(F_{1}).I(F_{2})+\overline{I(F_{1})}.\overline{I(F_{2})}$
\end{enumerate}
\end{proposition}
\begin{proof}
  Montrons le premier point de la proposition pr�c�dente, le point 2. est
  laiss� en exercice.
\begin{eqnarray*}
  I(F_{1}\to F_{2}) & = & I(F_2 \vee \neg F_{1}) 
\mbox{ par d�finition de }F_{1}\to F_{2}\\
  & = &I(F_{2}) + I(\neg F_{1}) \mbox{ par d�finition de }I(X \vee Y)\\
  & = & I(F_{2}) + \overline{I(F_{1})} \mbox{ par d�finition de }I(\neg X)\\
\mbox{donc }  I(F_{1}\vee F_{2}) & = & I(F_{2})+ \overline{I(F_{1})} 
\end{eqnarray*}
% \begin{enumerate}
% \item $I(F_{1}\wedge F_{2})=I(F_{1}).I(F_{2})$\begin{eqnarray*}
% I(F_{1}\wedge F_{2}) & = & I(\neg(F_{1}\to\neg F_{2}))\mbox{ par d�finition de }F_{1}\wedge F_{2}\\
%  & = & \overline{I(F_{1}\to\neg F_{2})}\mbox{ par d�finition de }I(\neg X)\\
%  & = & \overline{\overline{I(F_{1})}+I(\neg F_{2})}\mbox{ par d�finition de }I(X\to Y)\\
%  & = & \overline{\overline{I(F_{1})}+\overline{I(F_{2})}}\mbox{ par d�finition de }I(\neg X)\\
%  & = & \overline{\overline{I(F_{1})}}.\overline{\overline{I(F_{2})}}\mbox{ par les lois de De Morgan}\\
% I(F_{1}\wedge F_{2}) & = & I(F_{1}).I(F_{2})\end{eqnarray*}

% \item $I(F_{1}\opeq F_{2})=I(F_{1}).I(F_{2})+\overline{I(F_{1})}.\overline{I(F_{2})}$\begin{eqnarray*}
% I(F_{1}\opeq F_{2}) & = & I((F_{1}\to F_{2})\wedge(F_{2}\to F_{1}))\mbox{ par d�finition de }F_{1}\opeq F_{2}\\
%  & = & I(F_{1}\to F_{2}).I(F_{2}\to F_{1})\mbox{ par d�finition de }I(X\wedge Y)\\
%  & = & (\overline{I(F_{1})}+I(F_{2})).(\overline{I(F_{2})}+I(F_{1}))\mbox{ par d�finition de }I(X\to Y)\\
%  & = & \overline{I(F_{1})}.\overline{I(F_{2})}+\overline{I(F_{1})}.I(F_{1})+I(F_{2}).\overline{I(F_{2})}+I(F_{2}).I(F_{1})\mbox{ en d�veloppant}\\
% I(F_{1}\opeq F_{2}) & = & \overline{I(F_{1})}.\overline{I(F_{2})}+I(F_{2}).I(F_{1})\end{eqnarray*}
% \end{enumerate}
\end{proof}

\noindent \textbf{Exemple.}  Soit $F=\neg p\wedge((q\vee r)\to
s)$\begin{eqnarray*}
  I(F) & = & \overline{I(p)}.I((q\vee r)\to s)\\
  & = & \overline{I(p)}.(\overline{I(q\vee r)}+I(s))\\
  & = & \overline{I(p)}.(\overline{I(q)+I(r)}+I(s))\\
  I(F) & = &
  \overline{I(p)}.(\overline{I(q)}.\overline{I(r)}+I(s))\mbox{ d'apr�s
    les lois de De Morgan}
\end{eqnarray*} 
On retrouve le fait que si $I(p)=0,\; I(r)=1,\; I(q)=0$ et $I(s)=1$,
alors $I(F)=1$.  

%\clearpage
\begin{definition}
Soit $F$ une formule.
\begin{itemize}
\item $F$ est valide (ou une tautologie) si pour toute interpr�tation
  $I$, $I(F)=1$.
\item $F$ est satisfaisable s'il existe une interpr�tation $I$ telle
  que $I(F)=1$.
\item $F$ est non satisfaisable si pour toute interpr�tation $I$,
  I$(F)=0$.
\end{itemize}
% Un ensemble $\mathcal{F}$ de formules est satisfaisable s'il
% existe une interpr�tation $I$ qui satisfait toutes les formules $F$ de
% $\mathcal{F}$. Autrement dit, si $\F=\{F_1, \ldots, F_n\}$, alors $\F$
% est satisfaisable si la formule $\bigwedge_{i=1}^{n}F_{i}$ est
% satisfaisable, c'est-�-dire s'il existe $I$ telle que pour tout $i=1,
% \ldots, n$, $I(F_i)=1$.
\end{definition}
\noindent \textbf{Remarque 1.} Lorsqu'une interpr�tation $I$ est telle
que $I(F)=1$, on note parfois $I\vDash F$ qui se lit : \og $I$
satisfait $F$ \fg{} ou \og $F$ est vraie pour $I$ \fg{}.\\
\textbf{Remarque 2.}  Une formule $F$ est non satisfaisable si et
seulement si $\neg F$ est valide.

% \noindent \begin{center}
% \begin{pspicture}(0,-2.28)(3.36,2.28) \definecolor{color58}{rgb}{0.6,0.0,0.6} 
% \psellipse[linewidth=0.04,linecolor=color58,dimen=outer](1.68,0.0)(1.68,2.28) 
% \usefont{T1}{ptm}{m}{it} \rput(1.68,1.885){valide} \usefont{T1}{ptm}{m}{it} 
% \rput(1.69,0.4){satisfaisable,} \rput(1.69,-0.1){mais non valide} 
% \usefont{T1}{ptm}{m}{it} \rput(1.68,-1.055){non satisfaisable} 
% \psline[linewidth=0.04cm,linecolor=color58](0.14,0.86)(3.24,0.86) 
% \psline[linewidth=0.04cm,linecolor=color58](3.24,-0.74)(0.14,-0.74) 
% \usefont{T1}{ptm}{m}{n} \rput(1.71,1.265){$p\vee \neg p$} 
% \usefont{T1}{ptm}{m}{n} \rput(1.68,-1.535){$p\wedge \neg p$} 
% \end{pspicture} 
% \par\end{center}

\noindent \textbf{Exemple.}  Pour le probl�me de Kerstin, Pollet et
Anne, on consid�re la formule : $F= A \wedge B \wedge C$ (en omettant
les parenth�ses puisque l'interpr�tation de $\wedge$ est associative),
avec $A=p \to (q \wedge r)$, $B=\neg p\to q$ et $C= \neg p \to
r$. Donc pour toute interpr�tation $I$, on a : $I(A) = I(q)I(r) +
\overline{I(p)}$, $I(B) = I(q) + I(p)$ et $I(C) = I(r) + I(p)$. 
Le probl�me pos� revient � chercher les
interpr�tations $I$ pour lesquelles $F$ est vraie, c'est-�-dire
$I(F)=1$.\\ En posant $x= I(p)$, $y=I(q)$, $z=I(r)$, on obtient $I(F)
= (yz+\overline{x})(x+y)(x+z)$, on retrouve donc la fonction bool�enne
: $f(x,y,z)=(yz+\overline{x})(x+y)(x+z)$, pour laquelle on avait vu
que $f(x,y,z)=yz$. Donc $I(F)=1$ si et seulement si l'interpr�tation
$I$ est telle que $y=I(q)=1$ et $z=I(r)=1$. Ainsi, on en d�duit que
Anne et Pollet iront � la conf�rence mais qu'on ne sait pas pour
Kerstin.

\subsection{Equivalence s�mantique}
\begin{definition}
Deux formules $F$ et $G$ sont �quivalentes, not� $F\sim G$, si
pour toute interpr�tation $I$, on a : $I(F)=I(G)$.
\end{definition}
\noindent \textbf{Remarque 1.} La relation $\sim$ sur l'ensemble $CP$
des formules du calcul propositionnel est une relation d'�quivalence.\\
\textbf{Exemples.}  Les formules $p$ et $\neg(\neg p)$ sont
�quivalentes, et en g�n�ral :
\begin{itemize}
\item $F\sim\neg(\neg F)$
\item $F\vee G\sim G\vee F$ (car $+$ est commutatif dans $\Bo$)
\item $\neg(F\vee G)\sim(\neg F\wedge\neg G)$
\end{itemize}

\noindent \textbf{Remarque 2.}  On obtient des propri�t�s similaires �
l'associativit�, l'idempotence, l'absorption, la distributivit�, etc.
mais avec $\sim$ au lieu de l'�galit�. Ainsi, l'ensemble quotient
$CP/\sim$ est une alg�bre de Boole.\\

\noindent \textbf{Remarque 3.}  Sur $\mathcal{P}=\{ p_{1},\,\dots,\,
p_{n}\}$, une interpr�tation $I\colon\mathcal{P}\longrightarrow \Bo$
peut �tre identifi�e au $n$-uplet de ses valeurs
$I=(I(p_{1}),\,\dots,\, I(p_{n}))\in \Bo^n$. On peut donc associer �
toute formule $F$ sur $\mathcal{P}$ une fonction bool�enne
$g_{F}\colon \Bo^{n} \longrightarrow \Bo$ d�finie par $g_{F}(I)=I(F)$. 
Ainsi :
\begin{proposition}
  L'ensemble des fonctions bool�ennes est en bijection avec l'ensemble
  $CP/\sim$ (qui contient les formules du calcul propositionnel �
  �quivalence pr�s).
\end{proposition}
\noindent \textbf{Exemple.}  Soit $\begin{array}[t]{cccc}
  f\colon & \Bo^{2} & \longrightarrow & \Bo\\
  & (x,\, y) & \longrightarrow & \overline{x+y}\end{array}$\\
La formule $F$ de $CP$ qui lui est associ�e est $\neg (p \vee q)$
sur $\{ p,q\}$.\\
Cette correspondance permet d'associer � toute formule $F$ une formule
$F'$ sous forme normale conjonctive (FNC) et une formule $F''$ sous
forme normale disjonctive (FND), qui sont �quivalentes � $F$.
% \begin{definition}
%   Une clause est une formule de la forme $c_{1}\vee c_{2}\vee \ldots
%   \vee c_{n}$ o�, pour tout $i=1,\ldots,n$, ou bien
%   $c_{i}=p\in\mathcal{P}$, ou bien $c_i=\neg p\in\mathcal{P}$.
% \end{definition}
% On obtient alors :
% \begin{proposition}
% Toute formule est �quivalente � une conjonction de clauses (FNC).
% \end{proposition}

\subsection{Cons�quence s�mantique}
\begin{definition}
  Pour deux formules $F$ et $G$, on dit que $G$ est cons�quence de
  $F$, ou que $F$ satisfait $G$, not� $F\models G$ si pour toute
  interpr�tation $I$, si $I(F)=1$, alors $I(G)=1$.
\end{definition}
\begin{proposition}
$F$ satisfait $G$ ssi $(F\to G)$ est valide.
\end{proposition}
\begin{proof}
  Montrons l'�quivalence des n�gations, c'est-�-dire : $F$ ne
  satisfait pas $G$ ssi $(F\to G)$ n'est pas valide.

  $\bullet$ Si $F$ ne satisfait pas $G$, alors, par d�finition, il
  existe une interpr�tation $I$ telle que $I(F)=1$ et $I(G)=0$. Pour
  cette interpr�tation $I$, on a $I(F\to G)=\overline{I(F)}+I(G)=0$,
  donc $(F\to G)$ n'est pas valide.

  $\bullet$ R�ciproquement, si $(F\to G)$ n'est pas valide, alors il
  existe une interpr�tation $I$ telle que $I(F\to G)=0$, avec $I(F\to
  G)=\overline{I(F)}+I(G)$. Pour que la somme soit nulle, il faut que
  $I(F)=1$ et $I(G)=0$, donc $F$ ne satisfait pas $G$.

  \noindent Conclusion : $F$ satisfait $G$ ssi $(F\to G)$ est
  valide.
\end{proof}
\begin{proposition}
$F$ est �quivalente � $G$ ssi $F\opeq G$ est valide.
\end{proposition}

\noindent La d�monstration reprend le sch�ma pr�c�dent, elle est
laiss�e en exercice. On peut aussi v�rifier que $F \sim G$ ssi $F
\models G$ et $G \models F$ et utiliser la proposition pr�c�dente.

% \begin{proof}
%   Si $F\sim G$, alors pour toute interpr�tation $I$, $I(F)=I(G)$\\
%   Or par d�finition, $I(F\opeq
%   G)=I(F).I(G)+\overline{I(F)}.\overline{I(G)}$, donc $I(F\opeq
%   G)=I(F)+\overline{I(F)}=1$.  Ainsi, pour toute interpr�tation $I$,
%   $I(F\opeq G)=1$ et $F\opeq G$ est valide.

%   R�ciproquement, si $F\opeq G$ est valide, alors $I(F\opeq
%   G)=I(F).I(G)+\overline{I(F)}.\overline{I(G)}=1$ pour toute
%   interpr�tation $I$.\\  Supposons $I(F)\ne I(G)$. On aurait alors :
% \begin{itemize}
% \item Si $I(F)=1$, alors $I(G)=0$ et $I(F\opeq G)=0$
% \item Si $I(F)=0$, alors $I(G)=1$ et $I(F\opeq G)=0$
% \end{itemize}
% Ainsi, $I(F)=I(G)$ pour toute interpr�tation $I$ et donc $F\sim G$.
% \end{proof}

\medskip \noindent On �tend les d�finitions de la cons�quence
s�mantique $\models$ � des ensembles de formules.
\begin{definition}
  Soit $\mathcal{F}=\{ F_{1},\,\dots,\, F_{n}\} $ un ensemble fini de
  formules et $G$ une formule.
\begin{itemize}
\item On dit que $\mathcal{F}$ est satisfaisable s'il existe une
  interpr�tation $I$ telle que pour toute formule $F \in \mathcal{F}$,
  on ait $I(F)=1$. Donc $\mathcal{F}$ est satisfaisable si la formule
  $\bigwedge_{i=1}^{n}F_{i}$ est satisfaisable.
\item On note $\mathcal{F}\models G$ si
  $\bigwedge_{i=1}^{n}F_{i}\models G$, c'est-�-dire : pour toute
  interpr�tation $I$, si pour toute formule $F \in \F$, $I(F)=1$,
  alors $I(G)=1$.
\end{itemize}
\end{definition}
%\medskip{}
\noindent \textbf{Exemple.} Montrer que $\mathcal{H}=\{ p,\, p\to
q,\,\neg q\} $ n'est pas satisfaisable.

\begin{definition}
  Un s�quent est une paire $(\mathcal{F},\,G)$ o� $\mathcal{F}$ est un
  ensemble de formules et $G$ une formule.  Le s�quent
  $(\mathcal{F},\,G)$ est valide si $\mathcal{F}\models G$.
\end{definition}

\noindent \textbf{Exemple.}  Soient $\mathcal{F}=\{ p,\, p\to q\} $ et
$G=q$, on v�rifie que $(\mathcal{F},\,G)$ est un s�quent valide.

Montrons que pour toute interpr�tation $I$, si pour toute
$F\in\mathcal{F},\; I(F)=1$, alors $I(G)=1$.
Ceci revient � montrer $(p)\wedge(p\to q)\models q$.

Supposons $I(p)=I(p\to q)=1$. Or $I(p\to
q)=\overline{I(p)}+I(q)=1$. Mais comme $I(p)=1$, on a
$\overline{I(p)}=0$ et $I(q)=1$.  Ainsi, $(\mathcal{F},\,G)$ est un
s�quent valide.

\medskip \noindent \textbf{Remarque.}  Soit $\F'$ l'ensemble obtenu en
rempla�ant dans $\mathcal{F}$ un des $F_{i}$ par $F_{i}'$ tel que
$F_{i}\sim F_{i}'$. Alors : \\
- $\F$ est satisfaisable ssi $\F'$ est satisfaisable et \\
- $(\mathcal{F},\, G)$ est valide ssi $(\mathcal{F}',\, G)$ est
valide.

%\clearpage
\begin{proposition} Soit $\Hr$ un ensemble de formules, et $F$, $G$
  deux formules.  On a les �quivalences suivantes :
\begin{enumerate}
\item $\mathcal{H}\models G$ ssi $\mathcal{H}\cup\{ \neg G\} $
est non satisfaisable.
\item $\mathcal{H}\cup\{ F\} \models G$ ssi $\mathcal{H}\models(F\to G)$
\end{enumerate}
\end{proposition}
\begin{proof}
\begin{enumerate}
\item On d�montre l'�quivalence des n�gations. 
\begin{itemize}
\item Supposons $\mathcal{H}\cup\{ \neg G\} $ satisfaisable.\\
Alors il existe une interpr�tation $I$ telle que $I(F)=1$
pour toute formule $F$ de $\mathcal{H}$ et $I(\neg G)=1$, 
donc $I(G)=0$.\\
Donc on n'a pas $\mathcal{H}\models G$.
\item R�ciproquement, supposons que $\mathcal{H}\models G$ est
  faux. Alors il existe une interpr�tation $I$ telle que $I(F)=1$
  pour toute $F$ de $\mathcal{H}$
  et $I(G)=0$.\\
  Alors $I(\neg G)=1$ donc $I$ satisfait $\mathcal{H}\cup\{ \neg G\} $.
\end{itemize}
\item Montrons cette proposition � l'aide d'�quivalences.
\begin{eqnarray*}
  \mathcal{H}\cup\{ F\} \models G & \mbox{ssi} & \mathcal{H}\cup\{ F,\,\neg G\} \mbox{ non satisfaisable d'apr�s (1.)}\\
  & \mbox{ssi} & \mathcal{H}\cup\{ \neg(F\to G)\} \mbox{ non satisfaisable}\\
  & \mbox{ssi} & \mathcal{H}\models (F\to G)\mbox{ encore par (1.)}
\end{eqnarray*}
\end{enumerate}
\end{proof}

\subsection{Cons�quence logique (ou d�duction)}
\begin{definition}
  Un s�quent $\left(\mathcal{F},\, G\right)$ est dit prouvable, not�
  $\mathcal{F}\vdash G$, s'il est obtenu apr�s un nombre fini
  d'applications des six r�gles suivantes, o� $\Hr$ est un ensemble de
  formules, $F$ et $G$ des formules :
\begin{enumerate}

\item[(a)] Utilisation d'une hypoth�se : 
si $F\in \Hr$, alors $\mathcal{H}\vdash F$

\item[(b)] Augmentation d'hypoth�se : 
si $G\notin\mathcal{H}$ et $\mathcal{H}\vdash F$, 
alors $\mathcal{H}\cup \{ G\} \vdash F$

\item[(c)] Modus ponens :
si $\mathcal{H}\vdash\left(F\to G\right)$ et $\mathcal{H}\vdash F$,
alors $\mathcal{H}\vdash G$

\item[(d)] Retrait d'hypoth�se (synth�se) : 
si $\mathcal{H}\cup\left\{
    F\right\} \vdash G$, alors $\mathcal{H}\vdash\left(F\to G\right)$

\item[(e)] Double n�gation : $\mathcal{H}\vdash F$ ssi
  $\mathcal{H}\vdash\neg\neg F$

\item[(f)] Absurde : 
si $\mathcal{H}\cup\left\{ F\right\} \vdash G$ 
et $\mathcal{H}\cup\left\{ F\right\} \vdash\neg G$,
alors $\mathcal{H}\vdash\neg F$

\end{enumerate}\end{definition}
\noindent \textbf{Exemple.}
Preuve de la d�monstration par contrapos�e : 

On veut prouver : $p\to q\vdash\left(\neg q\to\neg p\right)$
\begin{enumerate}
\item $\left\{ p\to q,\,\neg q,\, p\right\} \vdash p$ {d'apr�s
a)}
\item $\left\{ p\to q,\,\neg q,\, p\right\} \vdash\neg q$ d'apr�s a)
\item $\left\{ p\to q,\,\neg q,\, p\right\} \vdash p\to q$ {d'apr�s
a)}
\item $\left\{ p\to q,\,\neg q,\, p\right\} \vdash q$ d'apr�s c) appliqu�e
� 1 et 3
\item $\left\{ p\to q,\,\neg q\right\} \vdash\neg p$ d'apr�s f) sur 2 et
4
\item $\left\{ p\to q\right\} \vdash\left(\neg q\to\neg p\right)$ d'apr�s
d)
\end{enumerate}

\begin{theorem}
Un s�quent $(\mathcal{F},\, G)$ est valide ssi il est prouvable.
\end{theorem}

\noindent \textbf{Remarque.}
Signification des deux sens de l'�quivalence :
\begin{itemize}
\item Sens $\Rightarrow$ : Compl�tude - \emph{Ce qui est vrai peut �tre
prouv�.}
\item Sens $\Leftarrow$ : Correction/ad�quation : \emph{Ce qui peut �tre
prouv� est vrai.}
\end{itemize}

\begin{proof}[Principe de la d�monstration de correction]
Par induction sur la longueur de la preuve.\\
A partir d'un s�quent valide, en appliquant une des six r�gles
a), ..., f) on obtient un nouveau s�quent valide.

Par exemple avec la r�gle a) :\\
On suppose que $\mathcal{F}\vdash G$ a �t� obtenu par la r�gle a),
donc : $G\in\mathcal{F}$.  Si $I$ est une interpr�tation telle que
$I(F)=1$ pour toute formule $F$ de $\mathcal{F}$, alors $I(G)=1$
puisque $G\in\mathcal{F}$, donc $\mathcal{F}\models G$ et le s�quent
est valide.
\end{proof}


%\clearpage
\section{Logique du premier ordre}
La logique du premier ordre enrichit le calcul propositionnel en utilisant :
\begin{itemize}
\item des termes construits avec des variables et des fonctions,
\item des formules construites � partir de relations sur les termes,
  avec des op�rateurs bool�ens et des quantifications sur les
  variables.
\end{itemize}
Par exemple $F: \ \forall x \exists y R(x,y)$ est une formule de la
logique du 1er ordre. Dans cette formule, $x$ et $y$ sont des
variables et $R(x,y)$ est une formule atomique construite en utilisant
une relation binaire $R$.
\begin{itemize}
\item Si $F$ est interpr�t�e sur les entiers naturels, avec pour $R$ la
  relation $<$, on obtient : \emph{Pour tout entier $x$, il existe un
    entier $y$ strictement plus grand que $x$}, ce qui exprime que
  l'ensemble des entiers naturels n'a pas d'�l�ment maximal.
\item Si $F$ est interpr�t�e sur l'ensemble des personnes avec pour
  $R$ la relation d�fnie par $R(x,y)$ si $y$ est la m�re de $x$, la formule
  exprime que \emph{toute personne a une m�re}. 
\item Si $F$ est interpr�t�e dans les mondes de Tarski, avec pour $R$
  la relation $LeftOf$, elle exprime que tout objet est � gauche d'un
  autre objet.
\end{itemize}

\subsection{Syntaxe}
On consid�re un ensemble $\G$ de symboles de fonctions et un ensemble
$\Rel$ de symboles de relations.  En particulier, on notera :
\begin{itemize}
\item $\C = \G_0$ l'ensemble des symboles de fonctions sans argument,
  c'est-�-dire les constantes,
\item $\Pa = \Rel_0$ l'ensemble des symboles de relations d'arit�
  nulle, c'est-�-dire les propositions (du calcul propositionnel), qui
  seront interpr�t�es dans $\Bo = \{0,1\}$.
\end{itemize}
On consid�re aussi un ensemble $X$ de variables et on d�finit les
termes et les formules de la logique du premier ordre associ�s � $\G
\cup \Rel \cup X$.
\begin{definition}[Termes avec variables] L'ensemble $T(\G,X)$ des
  termes est d�fini inductivement par :
\begin{itemize}
\item[(B)] toute constante de $\C$ est un terme et toute variable de
  $X$ est un terme,
\item[(I)] si $f \in \G$ a $n$ arguments et si $t_1, \ldots, t_n$ sont
  des termes, alors $f(t_1, \ldots, t_n)$ est un terme.
\end{itemize}
%Les termes sans variables sont appel�s des termes \emph{clos}.
\end{definition}

\begin{definition}[Formules] Les formules de la logique du premier
  ordre sur $\G$ et $\Rel$ sont d�finies inductivement par :
\begin{itemize}
\item[(B)] si $R \in \Rel$ a $n$ arguments et si $t_1, \ldots, t_n$
  sont des termes, alors $R(t_1, \ldots, t_n)$ est une formule dite
  \emph{atomique},
\item[(I)] si $F$ et $G$ sont des formules alors $\neg F$, $(F \wedge
  G)$, $(F \vee G)$, $(F \to G)$ sont des formules,\\
  Si $x$ une variable alors $\forall x F$ et $\exists x F$ sont des
  formules.
\end{itemize}
\end{definition} 

\noindent Par exemple, dans la formule $\forall x \forall y
(R(f(x,y),a) \to (R(x,a) \wedge R(y,a)))$, on trouve les termes $a$
(constante), $x$, $y$ (variables) et $f(x,y)$, pour une fonction $f$ �
deux arguments, ainsi qu'une relation binaire $R$.  On pourra par la
suite interpr�ter cette formule dans $\N$, avec l'addition pour $f$,
l'�galit� pour $R$, et la valeur $0$ pour $a$.

\medskip \noindent \textbf{Remarque 1.}  Les formules du calcul
propositionnel sont des cas particuliers de cet ensemble.

\medskip \noindent \textbf{Remarque 2.} Les formules 
peuvent �tre repr�sent�es par des arbres. 

%Par exemple, la formule $F : \ \forall z R(x,z) \to \exists y (R(y,z)
%\wedge Q(y,z))$ est repr�sent�e par l'arbre :

% \begin{center} 
% \begin{picture}(90,70)(0,0)
%   \gasset{Nframe=n,AHnb=0}

% \node(E1)(35,65){$\to$}  
% \node(E2)(15,50){$\forall z$}  \node(E3)(55,50){$\exists y$}  
% \node(E4)(15,35){$R$}  \node(E5)(55,35){$\wedge$}  
% \node(E6)(5,20){$x$} \node(E7)(25,20){$z$} 
% \node(E8)(40,20){$R$} \node(E9)(70,20){$Q$}
% \node(E10)(30,5){$y$} \node(E11)(50,5){$z$}
% \node(E12)(60,5){$y$} \node(E13)(80,5){$z$}

%  \drawedge(E1,E2){} \drawedge(E1,E3){}  
% \drawedge(E2,E4){} \drawedge(E3,E5){} 
% \drawedge(E4,E6){} \drawedge(E4,E7){} 
% \drawedge(E5,E8){} \drawedge(E5,E9){} 
% \drawedge(E8,E10){} \drawedge(E8,E11){} 
% \drawedge(E9,E12){} \drawedge(E9,E13){} 

% \end{picture}
% \end{center} 
%\clearpage
\subsection{Variables libres et li�es}

\begin{definition}[Variables d'un terme ou d'une formule]~\\
\vspace*{-0.5cm}
\begin{enumerate}
\item Les variables d'un terme sont d�finies inductivement par :
\begin{itemize}
\item $Var(x)=\{x\}$ si $x\in X$ et $Var(c) = \emptyset$ si $c \in \C$,
\item $Var(f(t_1, \ldots, t_n)) = \bigcup_{i=1}^{n} Var(t_i)$ pour un terme 
$f(t_1, \ldots, t_n)$.
\end{itemize}
\item Les variables d'une formule sont d�finies inductivement par :
\begin{itemize}
\item $Var(R(t_1, \ldots, t_n)) = \bigcup_{i=1}^{n} Var(t_i)$ pour une
  formule atomique,
\item si $F$ et $G$ sont des formules, $x$ est une variable et $* \in
  \{\wedge, \vee, \to \}$, alors : $Var(\neg F) = Var(F)$,
  $Var(F*G) = Var(F) \cup Var(G)$, $Var(\exists x F)= Var(\forall x F)
  = Var(F) \cup \{x\}$.
\end{itemize}
\end{enumerate}
\end{definition} 

\noindent \textbf{Remarque.} Les variables et les constantes n'ayant
pas d'argument, elles n'ont pas de descendant et sont toujours en
position de feuilles dans l'arbre associ� � une formule.

\begin{definition}[Variables libres et li�es]~\\
\vspace{-0.4cm}
\begin{itemize}
\item Dans l'arbre d'une formule $F$, une feuille
  d'�tiquette $x\in X$ est une occurrence \emph{libre} de $x$ s'il n'y
  a aucun quantificateur $\forall x$ ou $\exists x$ dans les
  ascendants de cette feuille. \\  Sinon, l'occurrence est dite
  \emph{li�e}.
\item Une variable est \emph{libre} dans une formule $F$ si elle a
  \textbf{au moins} une occurrence libre dans cette formule. Elle est
  \emph{li�e} dans une formule si elle n'est pas libre dans cette
  formule.
\end{itemize}
\end{definition}

On note $L(F)$ l'ensemble des variables libres dans $F$ et $B(F)=
Var(F) \setminus L(F)$ l'ensemble des variables li�es dans $F$ ($B$
pour \emph{bound} en anglais).

Par exemple, dans la formule $F$ ci-dessus, en consid�rant les
feuilles de gauche � droite, l'occurrence de $x$ est libre,
l'occurrence de $z$ est li�e, puis les deux occurrences de $y$ sont
li�es tandis que les deux occurrences de $z$ sont libres.\\
Par cons�quent, $L(F) =\{ x,\, z \} $ et $B(F)=\{y\}$ 

\begin{definition}[Formule close]
Une formule est dite \emph{close} si elle n'a aucune variable libre.
\end{definition}


\begin{proposition}
  Les variables libres d'une formule sont d�finies inductivement par :
\begin{itemize}
\item[$(B)$]  $L(R(t_{1},\,\dots,\, t_{n}))=\bigcup_{i=1}^{n}Var(t_{i})$
\item[$(I)$]  Pour $F$ et $G$ deux formules, 
\begin{itemize}
\item $L(\neg F)=L(F)$ et $L(F\star G)=L(F)\cup L(G)$
pour $\star\in\{ \wedge,\,\vee,\,\to\} $
\item $L(\forall xF)=L(\exists xF)=L(F) \setminus \{ x\} $
\end{itemize}
\end{itemize}
\end{proposition}


\subsection{S�mantique}

Pour interpr�ter les formules, on va consid�rer une structure
$\mathcal{M}$, donn�e par :
\begin{itemize}
\item un domaine $D$,
\item pour toute fonction $f$ de $\mathcal{G}$ � $n$ arguments, une
  fonction
  $f_{D}\colon D^{n}\longrightarrow D$
\item Pour toute relation $R$ de $\mathcal{R}$ � $n$ arguments, une
  relation $R_{D}\subseteq D^{n}$
\end{itemize}
En particulier, une constante $a$ correspond � un �l�ment $a_{D}$ de
$D$ et une proposition $p$ (relation sans argument) correspond � un
�l�ment de $\Bo=\{ 0,\,1\}$.

On note $\mathcal{M}=(D,\,(f_{D})_{f\in\mathcal{G}},
\,(R_{D})_{R\in\mathcal{R}})$ une telle structure.

\medskip \noindent \textbf{Exemple 1.}  Dans la structure $\M = (\R,
0, 1, +, \times, =)$, le domaine est $D=\R$, l'ensemble des nombres
r�els, les fonctions sont : les constantes $0$ et $1$, l'addition, la
multiplication et il y a un seul pr�dicat qui est l'�galit�.\\
Le terme $((x\odot x)\oplus x)\oplus a$ peut �tre interpr�t� sur ce
domaine, avec $a_{\R} = 1$, $\oplus_{\R}$ est l'addition, $\odot_{\R}$
est la multiplication. Il repr�sente donc le polyn�me $P$ d�fini par
$P(x)=x^{2}+x+1$.\\
Consid�rons maintenant la formule atomique $F\colon Q(((x\odot
x)\oplus x)\oplus a,\, b)$. En interpr�tant le pr�dicat binaire $Q$
comme l'�galit� et avec $b_{\R} = 0$, cette formule s'interpr�te comme
un pr�dicat unaire avec $x$ comme variable libre: $P(x) = 0$.\\
La formule $\exists x F$ correspond alors � l'�nonc� : le
polyn�me $P$ a une racine dans $\R$.

\medskip \noindent \textbf{Exemple 2.}  Lorsque $\mathcal{M}$ d�crit
une base de donn�es, les requ�tes sont des formules.

\begin{definition}
  Pour une structure $\mathcal{M}$ avec domaine $D$, une valuation
  est une application $v\colon X\longrightarrow D$.
\end{definition}
\begin{proposition}
  Etant donn�es une structure $\mathcal{M}$ associ�e �
  $\mathcal{G}\cup\mathcal{R}$, avec domaine $D$, et une valuation
  $v\colon X\longrightarrow D$, la valeur d'un terme $v^{*}(t)\in D$
  est d�finie inductivement par :
\begin{itemize}
\item[($B$)] $v^{*}(a)=a_{D}$ pour une constante $a$ et 
$v^{*}(x)=v(x)$ pour une variable $x\in X$,
\item[($I$)] Si $t=f(t_{1},\,\dots,\, t_{n})$ pour une fonction $f$ �
  $n$ arguments et des termes $t_{1},\,\dots,\, t_{n}$, alors
  $v^{*}(t)=f_{D}(v^{*}(t_{1}),\,\dots,\, v^{*}(t_{n}))$.
\end{itemize}
\end{proposition}

\noindent \textbf{Remarque.} A chaque $v\colon X\longrightarrow\mathcal{D}$,
on associe  $\begin{array}[t]{cccc}
v^{*}\colon & T(\mathcal{G},\, X) & \longrightarrow & \mathcal{D}\\
 & t & \longmapsto & v^{*}(t)\end{array}$
\begin{definition}
  Soient $\M$ une structure de domaine $D$, $v\colon X\longrightarrow
  D$ une valuation, $x\in X$ une variable et
  $a_{\mathcal{D}}\in\mathcal{D}$.\\
  La valuation $v'=v[x\mapsto a_{D}]$ est d�finie par : $\begin{cases}
    v'(y)=v(y) & \mbox{si }y\ne x\\
    v'(x)=a_{D} & \end{cases}$
\end{definition}

\medskip \noindent \textbf{Exemple.}
On consid�re $X=\{ x,\, y,\, z, \,w\} $ et $D=\N$. \\
D�terminer $v_{1}=v[z\to 3]$ et $v_{2}=v_{1}[x\to 1]$ pour la
valuation $v$ d�finie par :
\[ 
v\colon \left \{ 
\begin{array}{c}
x\mapsto2\\
y\mapsto8\\
z\mapsto7\\
w\mapsto14\end{array} \right.
\]

\begin{definition}
  On d�finit la valuation (ou valeur de v�rit�, ou interpr�tation)
  d'une formule $F$, not�e $\hat{v}(F)$, inductivement par :
\begin{itemize}
\item[$(B)$] Si $F=R(t_{1},\,\dots,\, t_{n})$, alors $\hat{v}(F)=1$
ssi $(v^{*}(t_{1}),\,\dots,\, v^{*}(t_{n}))\in R_{D}$
\item[$(B)$] Pour deux formules $F$ et $G$, et une variable $x$, 
\begin{itemize}
\item $\hat{v}(\neg F)=\overline{\hat{v}(F)}$
\item $\hat{v}(F\wedge G)=\hat{v}(F)\hat{v}(G)$
\item $\hat{v}(F \vee G)=\hat{v}(F)+\hat{v}(G)$
%
\item $\hat{v}(F \to G)=\overline{\hat{v}(F)}+\hat{v}(G)$
\item $\hat{v}(\forall x F)=1$ ssi pour toute $a_{D}\in D$,
  $\widehat{v[x\mapsto a_{D}]}(F)=1$
%
\item $\hat{v}(\exists x F)=1$ ssi il existe $a_{D}\in D$ tel que
  $\widehat{v[x\mapsto a_{D}]}(F)=1$
\end{itemize}
\end{itemize}
\end{definition}

\noindent \textbf{Exemple 1.} Soit $F: \ Q(((x\odot x)\oplus x)\oplus
a,\, b)$, la formule consid�r�e pr�c�demment, avec le m�me mod�le, et
la valuation $v$ telle que $v(x)=2$. Alors $\hat{v}(F)=0$.
De plus, comme aucune valeur de $x$ ne peut �tre racine, on a aussi 
$\hat{v}(\exists x F)=0$.

\medskip \noindent \textbf{Exemple 2.}  Soit $F=R(f(x),\, g(y))$ et la
structure $\mathcal{M}=(\N,\,f_{\N},\, g_{\N}, \, \leqslant)$, avec
$f_{\N}(n)=n+1$ et $g_{\N}(n)=n+3$, et soit $v_0$ la valuation d�finie 
par : $v_0\begin{cases}
x\mapsto4\\
y\mapsto3\end{cases}$
\begin{itemize}
\item $v_0^{*}(f(x))=f_{\N}(v_0(x))=f_{\N}(4)=5$
\item $v_0^{*}(g(y))=g_{\N}(v_0(y))=6$
\end{itemize}
Ainsi, $\hat{v_0}(F)=1$ car $5\leqslant6$ et, plus g�n�ralement, 
si $v$ est une valuation quelconque,  $\hat{v}(F)=1$ ssi $v(x)+1 \leq v(y)+3$.
%\clearpage
\begin{definition}
~\\
\vspace{-0.3cm} 
\begin{enumerate}
\item Etant donn�es une formule $F$ et une structure $\mathcal{M}$,
\begin{enumerate}
\item $F$ est satisfaisable pour $\mathcal{M}$, s'il existe une
  valuation $v$ telle que $\hat{v}(F)=1$.
\item $F$ est valide pour $\mathcal{M}$, si pour toute valuation $v$,
  $\hat{v}(F)=1$.\\
  On dit alors que $\mathcal{M}$ est un mod�le de $F$ (not�
  $\mathcal{M}\models F$).
\end{enumerate}
\item Etant donn�e une formule $F$ :
\begin{enumerate}
\item $F$ est satisfaisable s'il existe une structure $\mathcal{M}$ telle
que $F$ est satisfaisable pour $\mathcal{M}$.
\item $F$ est valide (ou universellement valide), si pour toute structure
$\mathcal{M}$, $F$ est valide pour $\mathcal{M}$.
\end{enumerate}
\end{enumerate}
\end{definition}

\noindent \textbf{Remarque.}  Le probl�me 2.a est ind�cidable. \\
Le probl�me 1.a est d�cidable pour des structures finies. C'est le cas
par exemple pour la satisfaisabilit� d'une requ�te dans une base de
donn�es.

\begin{proposition}
  Il existe un algorithme qui prend en entr�e une structure
  \emph{finie} $\mathcal{M}$ et une formule du premier ordre $F$, et
  qui d�cide s'il existe une valuation $v$ telle que $\hat{v}(F)=1$.
\end{proposition}

\noindent Pour les structures infinies, c'est plus compliqu� :
\begin{itemize}
\item Pour
  $\mathcal{M}_{1}=(\underbrace{\N}_{D},\,\underbrace{0, \,
    1,\,+,\,\times,\,\exp}_{\mathcal{G}},\,\underbrace{=}_{\mathcal{R}})$,
  la satisfaisabilit� (d'une formule du premier ordre) est
  ind�cidable.  \\
  Dommage... : $\exists n\exists x\exists y\exists
  z(x^{n}+y^{n}=z^{n}) \wedge (n\geqslant3)$

\medskip 
\item Pour $\mathcal{M}_{2}=(\N,\, 0, \, 1, \,+,\,\times,\,=)$, le
  probl�me est �galement ind�cidable.\\  
  Dommage... : $\exists xP(x)=0$ o� $P$ est un polyn�me.

\medskip

\item Mais pour $\mathcal{M}_{3}=(\R,\,+,\,\times,\,<)$, le probl�me
  est d�cidable !
\end{itemize}

\medskip
\noindent \textbf{Exemples de mod�les.}  Tout ensemble $E$ muni d'une
relation binaire $R_E$ r�flexive est un mod�le de la formule $F_1 : \
\forall x \ R(x,x)$.\\
Tout ensemble ordonn� $\M= (E, \preceq)$ est un mod�le de la formule
$F_1 \wedge F_2 \wedge F_3$ avec $F_1$ comme ci-dessus et :
\begin{itemize}
\item[] $F_2 : \ \forall x \ \forall y \ ((R(x,y) \wedge R(y,x)) \to x=y)$ 
\item[] $F_3 : \ \forall x \ \forall y \ \forall z \ ((R(x,y) \wedge R(y,z)) 
\to R(x,z))$
\end{itemize}


\subsection{Propri�t�s s�mantiques}
\begin{definition}
Soit $\mathcal{M}$ une structure, $F$ et $G$ deux formules.
\begin{itemize}
\item $F\models G$ si, pour toute valuation $v$ de $\mathcal{M}$, si
  $\hat{v}(F)=1$, alors $\hat{v}(G)=1$.  
\medskip
% \item Soit $\mathcal{F}=\{ F_{1},\,\dots,\, F_{n}\} $ un ensemble
% fini de formules et $G$ une formule.\\
% $\mathcal{F}\models G$ si ${\displaystyle \bigwedge_{i=0}^{n}F_{i}=F_{1}\wedge\cdots\wedge F_{n}\models G}$\\
% (ie. pour toute valuation $v$, si $\hat{v}(F)=1$
% pour toute $F\in\mathcal{F}$, alors $\hat{v}(G)=1$).
% \medskip
% \item Un s�quent est une paire $(\mathcal{F},\, G)$ o� $\mathcal{F}$
% est un ensemble fini de formules et $G$ une formule.\\
% Le s�quent $(\mathcal{F},\, G)$ est valide dans $\mathcal{M}$
% si $\mathcal{F}\models G$ dans $\mathcal{M}$.\\
% Il est universellement valide s'il est valide dans toute structure.
% \medskip
\item $F$ et $G$ sont �quivalentes, not� $F\sim G$ si
  pour toute structure $\mathcal{M}$ et pour toute valuation $v$,
  $\hat{v}(F)=\hat{v}(G)$
\end{itemize}
\end{definition}

\noindent \textbf{Remarques.}

\begin{itemize}
\item On note parfois $\models_{\M}$ au lieu de $\models$ lorsqu'il y
  a ambigu�t� sur la structure consid�r�e.
\item Comme pour le calcul propositionnel, $F\sim G$ ssi pour toute
  structure $\mathcal{M}$, on a : $F\models G$ et $G\models F$.
\item On a toutes les �quivalences du calcul propositionnel : \\
$\neg(F\vee G)\sim\neg F\wedge\neg G,\;\neg\neg F\sim F,\;\dots$
\item On voudrait des �quivalences plus \og riches \fg{}, qui impliquent
les variables.
\end{itemize}
 
\begin{proposition}
  Soit $F$ une formule.  On a l'�quivalence suivante :
\[ \neg\forall x F\sim\exists x\neg F\]
\end{proposition}
\begin{proof}[D�monstration]
Soit $\mathcal{M}$ une structure et $v$ une valuation. 
On a :
\begin{eqnarray*}
\hat{v}(\neg\forall xF)=1 & \mbox{ssi} & \hat{v}(\forall xF)=0\\
 & \mbox{ssi} & \mbox{il existe }a\in\mathcal{D}\mbox{ tq }\widehat{v[x\to a]}(F)=0\\
 & \mbox{ssi} & \mbox{il existe }a\in\mathcal{D}\mbox{ tq }\widehat{v[x\to a]}(\neg F)=1\\
 & \mbox{ssi} & \hat{v}(\exists x\neg F)=1\end{eqnarray*}
\end{proof}
\begin{lemma}
Soit $\mathcal{M}$ une structure, $v_{1}$ et $v_{2}$ deux valuations.
\begin{enumerate}
\item Si $t$ est un terme tel que $v_{1}|_{\mbox{Var}(t)} = v_{2}|_{\mbox{Var}(t)}$
($v_{1}$ et $v_{2}$ co�ncident sur $\mbox{Var}(t)$),\\
alors $v_{1}^{*}(t)=v_{2}^{*}(t)$
\item Si $F$ est une formule telle que $v_{1}|_{\mbox{L}(F)} = v_{2}|_{\mbox{L}(F)}$
($v_{1}$ et $v_{2}$ co�ncident sur les variables libres de $F$),\\
alors $\hat{v_{1}}(F)=\hat{v_{2}}(F)$
\end{enumerate}
\end{lemma}

\noindent \textbf{Exemple pour un terme.}  Soit $t=g(x,\, y)$ et deux
valuations $v_1$ et $v_2$ telles que $v_{1}(x)=v_{2}(x)$ et
$v_{1}(y)=v_{2}(y)$. \\ Alors $v_{1}^{*}(t)=g_{\mathcal{D}}(v_{1}(x),\,
v_{1}(y))=g_{\mathcal{D}}(v_{2}(x),\, v_{2}(y))=v_{2}^{*}(t)$.

\medskip \noindent \textbf{Remarque.}  Ce lemme exprime que la valeur
de v�rit� d'une formule ne d�pend que des valeurs de ses variables
libres. Par exemple, la valeur de la formule $F\colon\forall x R(y,\,
x)$ ne d�pend que de $y$.


\begin{corollary}[Cons�quences]
Si $F$ est une formule close (c'est-�-dire une formule sans variable libre), 
alors $\hat{v}(F)$ est constante, ind�pendante de $v$.

Ainsi, $F$ et $\forall xF$ sont �quivalentes si $x$ n'est pas libre
dans $F$.

De m�me, $F\sim\exists xF$ si $x\notin\mbox{L}(F)$.\end{corollary}

\noindent \textbf{Exemple.}
Soit $F\colon\forall x\exists y R(x,\, y)$ interpr�t�e avec pour
$\mathcal{D}$ l'ensemble des personnes et $R$ la relation d�finie par 
$R(x,y)$ si � $y$ est la m�re de $x$ �.
Alors, pour toute valuation $v$, $\hat{v}(F)=1$.


\medskip \noindent \textbf{Rappel :} D'apr�s la d�finition inductive de
$L(F)$, on a :\\
$L(\forall xF)=L(F)\backslash\{ x\} =L(F)$ si $x$ n'est pas libre dans $F$.

\noindent D�montrons la propri�t� : si
$x\notin L(F)$ alors $F\sim \forall x F$.\\

\noindent On se rappelle tout d'abord que, d'apr�s la d�finition
inductive de $L(F)$, on a : $L(\forall xF)=L(F)\backslash\{ x\} =L(F)$
si $x$ n'est pas libre dans $F$.

\noindent Soit maintenant $v$ une valuation. On a par d�finition de
$\hat{v}(\forall xF)$ :\\
$\hat{v}(\forall xF)=1$ si pour tout 
$a\in\mathcal{D},\,\widehat{v[x\to a]}(F)=1$.\\
Mais $v$ et $v[x\to a]$ co�ncident partout sauf sur $x$,
donc elles co�ncident sur $L(F)=L(\forall x F)$ d'apr�s ce qui pr�c�de.\\
Donc d'apr�s le lemme : $\hat{v}(F)=\widehat{v[x\to a]}(F)=1$, et 
on obtient bien $\hat{v}(F)=\hat{v}(\forall xF)$.
\iffalse

La formule $\forall x F$ est appel�e \emph{cl�ture universelle} de
$F$.
\begin{proposition}
Si $\mathcal{F}\models G$ dans $\mathcal{M}$ 
alors $\mathcal{F}\models\forall xG$
dans $\mathcal{M}$ si $x$ n'est libre dans aucune formule de $\mathcal{F}$
\end{proposition}
\vspace{-3mm}

\begin{proposition}[Substitution]
Soit $F$ une formule, $t$ un terme et $x$ une variable.\\
On veut remplacer toutes les occurrences libres de $x$ par $t$
dans $F\longrightarrow F[x\mapsto t]$ est la formule obtenue.
\begin{enumerate}
\item Cette op�ration est possible si aucune variable de $t$ ne se trouve
li�e dans $F(x\mapsto t)$.
\item On a alors pour toute structure $\mathcal{M}$ et toute valuation
$v$, $\hat{v}(F[x\mapsto t])=\widehat{v[x\mapsto v^{*}(t)]}(F)$
\end{enumerate}
\end{proposition}

\textbf{Exemple.}
Soit $F\colon\exists y(g(y,\, y)=x)$ interpr�t�e
dans $\N$ avec $g\to$ addition\\
$F$ dit que $x$ est pair.\\
On change la signification de la formule si on remplace $x$ par
$y$ : $\exists y(y+y=y)\longrightarrow y=0$ marche dans
$\N$.


\subsection{D�duction}
\begin{definition}
Un s�quent $(\mathcal{F},\, G)$ est dit prouvable s'il
est obtenu par application des r�gles a), b), c), d), e) et f) du
calcul propositionnel auxquelles on ajoute :

\vspace{-3mm}


\begin{enumerate}

\item Instantiation : si $\mathcal{F}\vdash\forall xF$, alors $\mathcal{F}\vdash F[x\mapsto t]$
(pour une substitution autoris�e)

\item Cl�ture universelle : si $\mathcal{F}\vdash F$ et si $x$
n'est pas libre dans $\mathcal{F}$, alors $\mathcal{F}\vdash\forall xF$

\item $\mathcal{F}\vdash\exists xF$ ssi $\mathcal{F}\vdash\neg\forall x\neg F$

\end{enumerate}\end{definition}
\begin{theorem}
Un s�quent est valide (universellement) ssi il est prouvable.\end{theorem}
\textbf{Exemple.}
On a : 
\begin{enumerate}
\item Prouver d'abord : $\mathcal{F},\, F\vdash G$ ssi $\mathcal{F},\, F\vdash\neg F$
\item Prouver ensuite : Si $\mathcal{F},\, F\vdash G$ et si $x$ n'est
pas libre dans $\mathcal{F}$ et $G$, alors $\mathcal{F},\,\exists xF\vdash G$
\item Prouver alors que : $\exists x\forall yF\vdash\forall y\exists xF$
\end{enumerate}
\fi









%%%%%%%%%%%%%%%%%%%%%%%%%%%%%%%%%%%%%%%
%\addcontentsline{toc}{section}{\numberline{}Bibliographie}
%\begin{small}
%\bibliography{phscomplete,tempsreel}
%\bibliographystyle{alpha}
%\end{small}

\end{document}

