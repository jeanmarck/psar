\documentclass[12pt,envcountsame]{article}


%\usepackage{epsf}
%\newcommand{\epsf}[1]{\epsfbox{#1}}

\usepackage{latexsym}
\usepackage{amssymb,amsmath}
\usepackage{amsthm,amsfonts}
\usepackage{gastex}
\usepackage[dvips]{graphics}
\usepackage{pstricks}
\usepackage{a4wide}
\usepackage{multicol}
\usepackage{pst-tree}
\usepackage{tikz}
\usepackage{multirow}

\usepackage[latin1]{inputenc}
%\usepackage[T1]{fontenc}
\usepackage[french]{babel}

\setcounter{tocdepth}{2}

\newcommand{\Q}{\mathbb{Q}}
\newcommand{\R}{\mathbb{R}}
\newcommand{\N}{\mathbb{N}}
\newcommand{\Z}{\mathbb{Z}}
\newcommand{\T}{\mathbb{T}}
\newcommand{\Bo}{\mathbb{B}}

\newcommand{\Au}{\mathcal{A}}
\newcommand{\B}{\mathcal{B}}
\newcommand{\C}{\mathcal{C}}
\newcommand{\D}{\mathcal{D}}
\newcommand{\F}{\mathcal{F}}
\newcommand{\G}{\mathcal{G}}
\newcommand{\K}{\mathcal{K}}
\newcommand{\Hr}{\mathcal{H}}
\newcommand{\I}{\mathcal{I}}
\newcommand{\La}{\mathcal{L}}
\newcommand{\M}{\mathcal{M}}
\newcommand{\Net}{\mathcal{N}}
\newcommand{\Pa}{\mathcal{P}}
\newcommand{\Rel}{\mathcal{R}}
\newcommand{\Sy}{\mathcal{S}}
\newcommand{\Ts}{\mathcal{T}}
\newcommand{\X}{\mathcal{X}}

\newcommand{\rel}{\bowtie}
\newcommand{\tr}{\xrightarrow}
\newcommand{\opeq}{\leftrightarrow}
\newcommand{\fee}{\varphi}
\newcommand{\eps}{\varepsilon}
\newcommand{\vect}[1]{\mathbf{#1}}
\newcommand{\fut}{\overrightarrow}
\newcommand\sui[1][a]{\ensuremath{\left(#1_n\right)_{n\in \N}}}

\newcommand{\E}{\textsf{E}}
\newcommand{\A}{\textsf{A}}

\DeclareMathOperator{\cla}{cl}

\theoremstyle{plain}

\newtheorem{theorem}{Th�or�me}
\newtheorem{lemma}[theorem]{Lemme}
\newtheorem{corollary}[theorem]{Corollaire}
\newtheorem{proposition}[theorem]{Proposition}

\newtheorem{definition}[theorem]{D�finition}

\theoremstyle{remark}
\newtheorem{remark}[theorem]{Remarque}
\newtheorem{example}[theorem]{Exemple}
\usetikzlibrary{arrows}
\tikzstyle{ent}=[circle,draw,thick,inner sep=0pt,minimum size=2.5mm]
\usetikzlibrary{arrows,automata,positioning}
\usetikzlibrary{automata,patterns,topaths,shapes,calc}
\tikzstyle{every picture}+=[>=stealth',initial text=]
\tikzstyle{state}=[rectangle,draw=black]

\begin{document}
\thispagestyle{empty}
\noindent UPMC


%\iffalse
%\vspace*{6cm}


\tableofcontents

%\vfill 
%\fi
%\input{ens-rel0}
%\input{rel-ordre}
%\input{induction}
%\input{auto}
\clearpage
\section{Introduction}  
  Le projet  \textbf{ "Interface graphique pour la Logique en L3"}consiste � d�velopper un outil dynamique et robuste pour am�liorer l'enseignement  de la logique en Licence 3. Ce projet nous a �t� soumis dans le cadre de L'UE PSAR du master 1 Informatique sp�cialit� SAR.
Il est sous la responsabilit� de Mr Fabrice Kordon suivit par Mme B�atrice Berard,Mr Mathieu Jaume et Mme B�n�dicte Legastelois. 

\section{Sp�cification Technique des Besoins et Exigences}
\subsection{Besoins et Exigences du client}
\subsubsection{Interpretation d'une formule de la logique des pr�dicats}
\noindent \textbf{Terme}
\\
Un terme est d�fini de fa�on inductive comme suit :
\begin{itemize}
\item Toute variable est un terme
\item Toute constantes est un terme
\end{itemize}
Nous manipulerons en tout cinq (5) variables \{v,w,x,y,z\}  et 20 constantes \{a,b,c,...,t\} dans notre projet.\\


\noindent \textbf{Interpr�tation des termes :}
\\
Interpr�ter un terme, c'est lui associer une valeur appartenant � un domaine. 
\begin{itemize}
\item Si le terme est une variable, c'est la valeur associ�e �  la variable
\item Sinon c'est la valeur globale r�sultant de l?�valuation des diff�rents arguments du terme\\
\end{itemize}


\noindent \textbf{Interpr�tation des termes du projet :}
\\
Dans ce projet les domaines d'interpr�tation seront des jardins  contenant des fleurs.

Les objets manipuler seront des fleurs dont les caract�ristiques sont les suivantes :
\begin{itemize}
\item esp�ce = {rose, paquerette, tulipe}
\item taille = {grand, moyen, petit}
\item couleur = {rouge, rose ,blanche}
\item nom{une constante ou none s'il est anonyme}
\end{itemize}
	
	
une fleur dans le jardin est repr�sent�e par un quintuplets \{(x,y),e,t,c,n\} o� :

\begin{itemize}
\item (x,y) est sa position dans le jardin 
\item e: son esp�ce
\item t: sa taille
\item c: sa couleur
\item n: son nom ou none si elle est anonyme
\end{itemize}

\newpage

\begin{center}
\textbf{Construction du jardin}
\end{center} 

les coordonn�es (x, y) sont des places du jardin 

\begin{figure}
\begin{center}
\begin{tikzpicture}[scale=0.375]
\node (n0)    at (-9.5,0)      {Ouest};
\node (n0)    at (9,0)      {Est};
\node (n0)    at (0,-9)      {Sud};
\node (n0)    at (0,9)      {Nord};

\node (n0)    at (0,1)      {$\bullet$};
\node (n0)    at (0,3)      {$\bullet$};
\node (n0)    at (0,5)      {$\bullet$};
\node (n0)    at (0,7)      {$\bullet$};
\node (n0)    at (0,-1)      {$\bullet$};
\node (n0)    at (0,-3)      {$\bullet$};
\node (n0)    at (0,-5)      {$\bullet$};
\node (n0)    at (0,-7)      {$\bullet$};

\node (n0)    at (1,0)      {$\bullet$};
\node (n0)    at (3,0)      {$\bullet$};
\node (n0)    at (5,0)      {$\bullet$};
\node (n0)    at (7,0)      {$\bullet$};
\node (n0)    at (-1,0)      {$\bullet$};
\node (n0)    at (-3,0)      {$\bullet$};
\node (n0)    at (-5,0)      {$\bullet$};
\node (n0)    at (-7,0)      {$\bullet$};

\node (n0)    at (1,2)      {$\bullet$};
\node (n0)    at (1,4)      {$\bullet$};
\node (n0)    at (1,6)      {$\bullet$};
\node (n0)    at (1,-2)      {$\bullet$};
\node (n0)    at (1,-4)      {$\bullet$};
\node (n0)    at (1,-6)      {$\bullet$};
\node (n0)    at (-1,2)      {$\bullet$};
\node (n0)    at (-1,4)      {$\bullet$};
\node (n0)    at (-1,6)      {$\bullet$};
\node (n0)    at (-1,-2)      {$\bullet$};
\node (n0)    at (-1,-4)      {$\bullet$};
\node (n0)    at (-1,-6)      {$\bullet$};

\node (n0)    at (2,1)      {$\bullet$};
\node (n0)    at (2,3)      {$\bullet$};
\node (n0)    at (2,5)      {$\bullet$};
\node (n0)    at (2,-1)      {$\bullet$};
\node (n0)    at (2,-3)      {$\bullet$};
\node (n0)    at (2,-5)      {$\bullet$};
\node (n0)    at (-2,1)      {$\bullet$};
\node (n0)    at (-2,3)      {$\bullet$};
\node (n0)    at (-2,5)      {$\bullet$};
\node (n0)    at (-2,-1)      {$\bullet$};
\node (n0)    at (-2,-3)      {$\bullet$};
\node (n0)    at (-2,-5)      {$\bullet$};

\node (n0)    at (3,0)      {$\bullet$};
\node (n0)    at (3,2)      {$\bullet$};
\node (n0)    at (3,4)      {$\bullet$};
\node (n0)    at (3,-2)      {$\bullet$};
\node (n0)    at (3,-4)      {$\bullet$};
\node (n0)    at (-3,0)      {$\bullet$};
\node (n0)    at (-3,2)      {$\bullet$};
\node (n0)    at (-3,4)      {$\bullet$};
\node (n0)    at (-3,-2)      {$\bullet$};
\node (n0)    at (-3,-4)      {$\bullet$};

\node (n0)    at (4,1)      {$\bullet$};
\node (n0)    at (4,3)      {$\bullet$};
\node (n0)    at (4,-1)      {$\bullet$};
\node (n0)    at (4,-3)      {$\bullet$};
\node (n0)    at (-4,1)      {$\bullet$};
\node (n0)    at (-4,3)      {$\bullet$};
\node (n0)    at (-4,-1)      {$\bullet$};
\node (n0)    at (-4,-3)      {$\bullet$};

\node (n0)    at (5,0)      {$\bullet$};
\node (n0)    at (5,2)      {$\bullet$};
\node (n0)    at (5,-2)      {$\bullet$};
\node (n0)    at (-5,0)      {$\bullet$};
\node (n0)    at (-5,2)      {$\bullet$};
\node (n0)    at (-5,-2)      {$\bullet$};

\node (n0)    at (6,1)      {$\bullet$};
\node (n0)    at (-6,1)      {$\bullet$};
\node (n0)    at (6,-1)      {$\bullet$};
\node (n0)    at (-6,-1)      {$\bullet$};

%\draw[dotted] (-8,8) --  (-8,-8);
\draw[dotted] (-7,8) --  (-7,-8);
\draw[dotted] (-6,8) --  (-6,-8);
\draw[dotted] (-5,8) --  (-5,-8);
\draw[dotted] (-4,8) --  (-4,-8);
\draw[dotted] (-3,8) --  (-3,-8);
\draw[dotted] (-2,8) --  (-2,-8);
\draw[dotted] (-1,8) --  (-1,-8);
\draw[->] (0,-8) --  (0,8);
\draw[dotted] (1,8) --  (1,-8);
\draw[dotted] (2,8) --  (2,-8);
\draw[dotted] (3,8) --  (3,-8);
\draw[dotted] (4,8) --  (4,-8);
\draw[dotted] (5,8) --  (5,-8);
\draw[dotted] (6,8) --  (6,-8);
\draw[dotted] (7,8) --  (7,-8);
%\draw[dotted] (8,8) --  (8,-8);

%\draw[dotted] (-8,8) --  (8,8);
\draw[dotted] (-8,7) --  (8,7);
\draw[dotted] (-8,6) --  (8,6);
\draw[dotted] (-8,5) --  (8,5);
\draw[dotted] (-8,4) --  (8,4);
\draw[dotted] (-8,3) --  (8,3);
\draw[dotted] (-8,2) --  (8,2);
\draw[dotted] (-8,1) --  (8,1);
\draw[->] (-8,0) --  (8,0);
\draw[dotted] (-8,-1) --  (8,-1);
\draw[dotted] (-8,-2) --  (8,-2);
\draw[dotted] (-8,-3) --  (8,-3);
\draw[dotted] (-8,-4) --  (8,-4);
\draw[dotted] (-8,-5) --  (8,-5);
\draw[dotted] (-8,-6) --  (8,-6);
\draw[dotted] (-8,-7) --  (8,-7);
%\draw[dotted] (-8,-8) --  (8,-8);

\draw[densely dashed] (0,0) -- (7,7);
\draw[densely dashed] (0,0) -- (-7,-7);
\draw[densely dashed] (0,0) -- (7,-7);
\draw[densely dashed] (0,0) -- (-7,7);
\end{tikzpicture}
\end{center}
\caption{Places d'un jardin}\label{lagrille}
\end{figure}

\[
\mathrm{Places}  = \left \{
\begin{array}{c}
(0,7) ,
\\
(-1,6), (1,6) ,
\\
(-2,5) , (0,5) , (2,5),
\\
(-3,4) , (-1,4) , (1,4) , (3,4),
\\
(-4,3), (-2,3), (0,3) , (2,3), (4,3),
\\
(-5,2),(-3,2) , (-1,2) , (1,2) , (3,2), (5,2),
\\
(-6,1),(-4,1), (-2,1), (0,1) , (2,1) , (4,1),(6,1),
\\
(-7,0) , (-5,0) , (-3,0) , (-1,0) , (1,0) , (3,0) , (5,0) , (7,0) ,
\\
(-6,-1),(-4,-1), (-2,-1), (0,-1) , (2,-1), (4,-1),(6,-1),
\\
(-5,-2),(-3,-2), (-1,-2), (1,-2) , (3,-2),(5,-2),
\\
(-4,-3), (-2,-3) , (0,-3) , (2,-3), (4,-3),
\\
(-3,-4), (-1,-4), (1,-4) , (3,-4) ,
\\
(-2,-5) , (0,-5) , (2,-5),
\\
(-1,-6),  (1,-6),
\\
(0,-7)
\\
\end{array}
\right \}
\]

\newpage	

\noindent \textbf{Formule de la logique des pr�dicats: }

Un Symbole de pr�dicat correspond au nom d'une propri�t� portant sur un ou plusieurs termes.

\emph{Exemple de pr�dicat:}\\
\begin{itemize}
\item[( f1 )] d est une rose :\\
 $\qquad Rose(d)$;
\item[( f2 )] Toutes les fleurs sont des roses :\\
 $ \qquad \forall x ~ Rose(x)$;
\item[( f3 )] Il existe une rose :\\
 $ \qquad \exists x ~ Rose(x)$; 
\item[( f4 )] Toute fleur blanche est plus petite qu'une fleur situ�e � son est :\\
$ \qquad \forall x ~ (est\_blanc(x) \implies  \exists y   ~ (plus\_petit\_que(x, y) \wedge a\_l\_est\_de(y, x)))$;
\item[( f5 )] Toute fleur est � l'est ou � l'ouest, ou au sud, ou au nord :\\
$ \qquad \forall x (a\_l\_est(x) \vee a\_l\_ouest(x)\vee au\_sud(x) \vee au\_nord(x))$;

\item[( f6 )] Toutes les grandes fleurs sont rouges et il n'existe pas de fleur blanche au sud d'une fleur rouge :\\
$ \qquad \forall x  ~ (est\_grand(x) \implies est\_rouge(x)) \wedge \not\exists x (est\_blanc(x) \wedge \exists y ~ (est\_rouge(y) \wedge au\_sud\_de(x, y)))$;

\item[( f7 )] Il existe une fleur rouge au nord de la fleur g :\\
$ \qquad \exists x ~ (est\_rouge(x) \wedge au\_nord\_de(x, g))$  \\
\end{itemize}


\noindent \textbf{Impl�mentation des formules}\\
Une variable est libre si elle n'est associ�e � aucun quantificateur (existentiel ,pour tout). Dans le cas du projet, les formules ne devront pas contenir des variables libres.
\\
\noindent \textbf{Formules du projet}\\
Il existe des pr�dicats unaire, binaires et ternaires :

\begin{itemize}
\item P1-pr�dicats unaire
\item P2- pr�dicats binaires
\item P3- pr�dicats ternaire
\end{itemize}
\\
P1 = \{est\_rouge, est\_rose, est\_blanc, a\_l\_est, a\_l\_ouest, au\_sud, au\_nord , Rose, Paquerette, Tulipe, est_grand, est_moyen, est_petit \}\\
P2 = \{a\_l\_est\_de, a\_l\_ouest\_de, au\_sud\_de, au\_nord\_de, m�me\_latitude, m�me\_longitude ,plus\_grand\_que,\\ plus\_petit\_que, m�me\_taille\_que, m�me\_couleur\_que\}\\
P3 = \{est\_entre \}

\begin{center}
\begin{tabular}{|l|c|r|}
\hline Pr�dicat & D�finition & Relation \tabularnewline
\hline Esp�ce & Rose, paquerette,Tulipe & Unaire\tabularnewline
\hline Taille & est\_grand,est\_moyen,est\_petit & Unaire \tabularnewline
\hline Couleur & est\_rouge,est\_rose,est\_blanc & Unaire\tabularnewline
\hline Position & a\_l\_est,a\_l\_ouest,au\_sud,au\_nord & Unaire \tabularnewline
\hline
Comparaison de position & a\_lest\_de,a\_l\_ouest\_de,au\_sud\_de & Binaire \tabularnewline
\hline & au\_nord\_de ,m�me\_latitude, m�me\_longitude &  \tabularnewline
\hline Comparaison de taille & Plus\_petit\_que,plus\_grand\_que,m�me\_taille\_que & Binaire\tabularnewline
\hline Comparaison de couleur & m�me\_couleur\_que & Binaire \tabularnewline
\hline �galit� & = & Binaire \tabularnewline
\hline Comparaison de position & est\_entre & Ternaire \tabularnewline
\hline
\end{tabular}
\end{center}


Toutes  les formules soumises � la v�rification doivent ne sont pas cens�es contenir des variables libres (une erreur est donc d�clench�e lors de l' �valuation d'une formule avec une variable libre).

\newpage

\section{Solutions}
\subsection{Architecture logicielle}
Le projet est structur� en modules ind�pendants : \\

\noindent \textbf{1-Module d'analyseur syntaxique des formules }\\
L'analyseur syntaxique sera un module du projet qui s'occupera de la v�rification syntaxique des diff�rentes formules qui seront soumises � une �valuation �ventuelle.\\
Une fois que  l?utilisateur (l'�tudiant) saisit sa formule et demande une interpr�tation, nous l'appellerons en premier pour s'assurer que la formule est bien form�e avant d'appeler notre interpr�teur.
\\L'analyseur syntaxique sera un module du projet qui s'occupera de la v�rification syntaxique des diff�rentes formules qui seront soumises � une �valuation �ventuelle.\\
Une fois que  l'utilisateur (l'�tudiant) saisit sa formule et demande une interpr�tation, nous l'appellerons en premier pour s'assurer que la formule est bien form�e avant d'appeler notre interpr�teur.\\

\noindent \textbf{2-Module Interface graphique}\\
Toute l'application sera pr�sent�e dans une interface conviviale, dans laquelle l'utilisateur aura la possibilit� de construire ses propres formules, son propre environnement et demander l'�valuation de ses formules.\\
Il est �vident que l'interface offrira un menu standard dans lequel il pourra sauvegarder/restaurer son environnement, ses formules dans/depuis le r�pertoire de son choix.\\
Notre choix du langage tient compte en premier lieu du temps de r�ponse et de la robustesse.\\



\noindent \textbf{3-Module �valuant les formules(Moteur)}\\
Nous disposons d'un module d'interpr�tation d�velopp� en python, il faudra faire  communiquer les deux modules pour obtenir l'interpr�tation de(s) la formule(s).\\
Le module python attend deux param�tres :\\

\begin{itemize}
\item l'environnement bien d�fini
\item la formule bien form�e 
\end{itemize}

\subsection{Choix technologique}

Comme le moteur  est d�velopp� en  python, nous partons de l?id�e de faire communiquer deux langages car, impl�menter une interface graphique en python s?av�re fastidieux.\\

\begin{center}
Resultats de nos recherches \\
\end{center} 

Notre application sera	 impl�ment�e en java, il faut trouver le middleware n�cessaire pour la communication des deux langages.\\
Le choix du middleware d�pend fortement :\\
\begin{itemize}
\item du temps de r�ponse
\item de la prise en main\\
\end{itemize}

\emph{Faire Communiquer java et python}\\

La Biblioth�que \textbf{Jython} permet d?ex�cuter en Java un programme �crit en python, Il existe aussi jeep qui fait de m�me.\\
\emph{D�savantage}: Il y a en effet une latence �lev�e de l?interpr�teur Jython.\\
\textbf{CORBA, ICE} et ses variantes sont de langage de d�finitions d?interfaces (IDL)  visant � faire communiquer deux entit�s (services, composants) �crit dans des langages diff�rents; Il s?agit de mapper (traduire des �l�ments fournis par l?IDL en �l�ments d?un langage de programmation).\\
Nous utiliserons cette deuxi�me approche dans le cadre de ce projet.

\subsection{D�lais et taches � r�aliser}
L'ue PSAR est organis� comme suit :
\begin{itemize}
\item R�daction d'un cahier de charge
\item R�alisation du projet
\item R�daction d'un rapport
\item Soutenance du projet
\end{itemize}

Comme tout projet cela est g�r� dans un temps bien d�finis.
\begin{itemize}
\item 01 Mars 2016: Cahier de charges
\item Rapport
\item Soutenance du projet
\end{itemize}


\subsection{Organisation et gestion du projet}

Nous avons d�cid� pour chaque module de discuter et partager nos id�es, avant de nous r�partir le travail. Ensuite, apr�s que chacun de nous ai fini, de travailler sur un composant de l?application, ou sur une classe, nous  passerons � la mise en commun et � la r�alisation des diff�rents tests � diff�rent niveau d?abstraction.\\

\noindent \textbf{Gestion du projet}\\
 Dans le cadre du projet, afin de nous assurer de g�rer au mieux les incoh�rences et les bugs, nous utiliserons l?outil Git.
\\
\noindent \textbf{Gestion des documents dans le projet} \\
Tous les documents que nous produirons dans le cadre du projet (cahier de charges, rapport d'avancement,Rapport final) serons �dit�s en latex.




%%%%%%%%%%%%%%%%%%%%%%%%%%%%%%%%%%%%%%%
%\addcontentsline{toc}{section}{\numberline{}Bibliographie}
%\begin{small}
%\bibliography{phscomplete,tempsreel}
%\bibliographystyle{alpha}
%\end{small}

\end{document}

